

\documentclass{amsart} 
\usepackage{scrextend, amssymb,amsmath,latexsym,times,tikz,hyperref,mathrsfs,enumitem,soul,pgfplots, listings}
%\usepackage{showlabels}
\setstcolor{red}
\setul{0pt}{2pt}

\lstset{
  language=Python,
  aboveskip=3mm,
  belowskip=3mm,
  showstringspaces=false,
  columns=flexible,
  basicstyle={\small\ttfamily},
  numbers=none,
  numberstyle=\tiny\color{gray},
  keywordstyle=\color{blue},
  commentstyle=\color{dkgreen},
  stringstyle=\color{mauve},
  breaklines=true,
  breakatwhitespace=true,
  tabsize=3
}

\hypersetup{colorlinks=true, linkcolor=blue, citecolor=magenta}

\newcommand{\btu}{\bigtriangleup}

\numberwithin{equation}{section}

\theoremstyle{plain}
\newtheorem{thm}{Theorem}[section]
\newtheorem{dfn}[thm]{Definition}
\newtheorem{cor}[thm]{Corollary}
\newtheorem{prop}[thm]{Proposition}
\newtheorem{lemma}[thm]{Lemma}
\newtheorem{obs}[thm]{Observation}
\newtheorem{step}[thm]{}
\newtheorem{conj}[thm]{Conjecture}
\newtheorem{sublemma}{}[thm]
\newtheorem{question}[thm]{Question}
\newtheorem{answer}[thm]{answer}
\newtheorem{explanation}{Explanation}


\newtheorem*{lemma*}{Lemma}

\newcommand{\mgn}[1]{\marginpar{\begin{flushleft}\footnotesize
      \textcolor{red}{#1}\end{flushleft}}}
\newcommand{\mgnb}[1]{\marginpar{\begin{flushleft}\footnotesize
      \textcolor{blue}{#1}\end{flushleft}}}
\newcommand{\ba}{\backslash}

\theoremstyle{definition}
\newtheorem{example}{Example}

\newcommand{\del}{\backslash}
\DeclareMathOperator{\cl}{cl}
\DeclareMathOperator{\cy}{cy}

\DeclareMathAlphabet{\mathdutchcal}{U}{dutchcal}{m}{n}
\usetikzlibrary{calc}
\title{Proofs/Notes}
\author{Nikhil Vasan}
\date{September 2021}

\begin{document}
\maketitle
\section{Overview}
This section describes the modifications proposed for the Canto implementation of the EIP-1559. As opposed to the 
evmos implementation of EIP-1559, we intend to burn the baseFeePerBlock * gasUsed 
\section{Implementation}    
The implementation of the burning of the baseFee will be done through introducing a dependency between the Distribution Keeper, and feeMarket Keeper as such, in x/Distribution/types/interfaces.go
\begin{lstlisting}
      //In x/distribution/types/interfaces.go
      type FeeMarketKeeper interface{ 
            GetBaseFee(ctx sdk.Context) *big.Int
            GetBlockGasWanted(ctx sdk.Context) uint64
      }
\end{lstlisting}

\section{Math}

\begin{thm}
      
\end{thm}


\end{document}